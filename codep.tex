

\documentclass[11pt]{article}

\usepackage{drl-common/diagrams}
\usepackage{multicol}
\usepackage{color}
\usepackage{amsmath}
\usepackage{amssymb}
\usepackage{amsthm}
\usepackage{stmaryrd}
\usepackage{drl-common/proof}
\usepackage{drl-common/typesit}
\usepackage{drl-common/typescommon}
\usepackage{drl-common/theorem-envs}
\usepackage{graphics}
\usepackage{url}
\usepackage{relsize}
\usepackage{fancyvrb}
\usepackage{tikz}
\usetikzlibrary{decorations.pathmorphing}

\usepackage{drl-common/code}
\DefineVerbatimEnvironment{code}{Verbatim}{fontsize=\small,fontfamily=tt}


\title{Codependent Types}

\author{DRL \and PLL \and ER \and MS}

\begin{document}

\maketitle

\section{Cofibrations}

\newcommand\complex[1]{#1 \, \dsd{complex}}
\newcommand\wfcell[2]{#1 \rightarrowtail #2 \, \dsd{cell}}
\newcommand\compose[3]{\kappa \, \tptmns{#1}{#2}. {#3}}
\newcommand\tsubst[2]{#1 [ #2 ]}
\newcommand\wfsub[3]{#1 \dashv #2 : #3}
\newcommand\inc{\iota}
\newcommand\idsub{\dsd{id}}
\newcommand\esub[3]{#1 , (#2 \mapsto #3)}
\newcommand\one{\dsd{1}}
\newcommand\triv{()}
\newcommand\ofret[3]{#1 \dashv #2 \, : \, #3}
\newcommand\inco[1]{\inc_1(#1)}
\newcommand\inct[1]{\inc_2(#1)}
\newcommand\emptyc[0]{\emptyset}

\framebox {$\complex \Psi$}  

\[
\infer{\complex \emptyc}{}
\qquad
\infer{\complex{\Psi,x:C}}
      {\complex \Psi & 
        \wfcell \Psi C}
\]

\framebox {$\wfcell \Psi C$} Represents a cofibration $\Psi \rightarrowtail \Psi.C$

\[
\infer{\wfcell {\Psi_1}{\tsubst{C}{\theta}}}
      {\wfcell {\Psi_2}{C} &
       \wfsub {\Psi_1} {\theta} {\Psi_2}}
\qquad
\infer{\wfcell {\Psi} \one}{}
\qquad
\infer{\wfcell {\Psi} {\compose{x}{C_1}{C_2}} }
      {\wfcell \Psi {C_1} & 
        \wfcell {\Psi,\tptm{x}{C_1}} {C_2}}
\]

Base types like
\[
\infer{\wfcell {\tptm{x}{\one},\tptm{y}{\one}} {I(x,y)} }
      {}
\]

\framebox {$\wfsub \Psi \theta \Psi'$} Represents a map \emph{from} $\Psi'$ \emph{to} $\Psi$.  

\[
\infer{\wfsub{\Psi}{[]}{\emptyc}}{}
\qquad
\infer{\wfsub{\Psi}{\esub \theta x c}{\Psi',\tptm{x}{A}}}
      {\wfsub{\Psi}{\theta}{\Psi'} & 
       \oftp{\Psi'}{c}{\tsubst{C}{\theta}}
      }
\qquad
\]

Admissible:
\[
\infer{{\wfsub{\Psi,\tptm{x}{C}}{\inc}{\Psi}}}{}
\qquad
\infer{\wfsub{\Psi}{\idsub}{\Psi}}{}
\qquad
\infer{\wfsub{\Psi_1}{\tsubst{\theta_2}{\theta_1}}{\Psi_3}}
      {\wfsub{\Psi_1}{\theta_1}{\Psi_2} & 
        \wfsub{\Psi_2}{\theta_2}{\Psi_3} & 
      }
\]

\framebox {$\ofret \Psi c C$} Represents a retraction of $\Psi.C$ onto
$\Psi$, i.e. a retract of $\Psi \rightarrowtail \Psi.C$.  

\[
\infer{\ofret{\Psi,\tptm{x}{C},\Psi'}{x}{C}}{}
\qquad
\infer{\ofret {\Psi_1}{\tsubst{c}{\theta}}{\tsubst{C}{\theta}}}
      {\ofret {\Psi_2}{c}{C} &
       \wfsub {\Psi_1} {\theta} {\Psi_2}}
\]

\[
\infer{\oftp \Psi {\triv} \one}{}
\qquad
\infer{\ofret {\Psi}{[c_1,c_2]}{\compose{x}{C_1}{C_2}}}
      {\ofret {\Psi}{c_1}{C_1} &
        \ofret {\Psi}{c_2}{C_2[c_1/x]} &
      }
\]

\[
\infer{\ofret {\Psi}{\inco{c}}{C_1}}
      {\ofret {\Psi}{c}{\compose{x}{C_1}{C_2}}}
\qquad
\infer{\ofret {\Psi}{\inct{c}}{C_2[\inco{c}/x]}}
      {\ofret {\Psi}{c}{\compose{x}{C_1}{C_2}}}
\]

\subsection{Examples}

\newcommand\II[0]{\mathbb{I}}

The endpoint inclusions of the interval:
\[
\begin{array}{l}
\II := \compose{x}{\one}{\compose{y}{\one}{I(x,y)}}\\
\ofret{\tptm{x}{\II}}{\inco{x}}{\one}\\
\ofret{\tptm{x}{\II}}{\inco{\inct{x}}}{\one}
\end{array}
\]

\end{document}
